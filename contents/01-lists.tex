% ==========================================================
\section{Lists and Enumerations (列表与枚举)}
% ==========================================================

\subsection{Unordered Lists (无序列表)}

\subsubsection{功能介绍 (Introduction)} 

无序列表用于展示没有特定顺序的项目集合。默认使用圆点作为符号,但可以通过 \texttt{enumitem} 宏包进行高度定制。

Unordered lists are used to display a collection of items with no specific order. They use bullets by default but can be highly customized using the \texttt{enumitem} package.

\vspace{0.5em}
\subsubsection{代码展示 (Code)}

\begin{minted}{latex}
\begin{itemize}
    \item Default item
    \item[--] Custom dash item
    \item[\textcolor{red}{$\bullet$}] Red bullet item
\end{itemize}
\end{minted}

\subsubsection{语法详解 (Syntax)}

\begin{itemize}
    \item \code{\textbackslash begin\{itemize\}}: 开启无序列表环境。
    \item \code{\textbackslash item}: 开始一个新列表项。
    \item \code{[label]}: 可选参数,用于覆盖默认符号 (需 \texttt{enumitem})。
    \item \code{\textbackslash bullet}: 默认的圆点符号。
\end{itemize}

\subsubsection{效果展示 (Effect)}

\begin{itemize}
    \item Default item
    \item[--] Custom dash item
    \item[\textcolor{red}{$\bullet$}] Red bullet item
\end{itemize}

% ----------------------------------------------------------

\subsection{Ordered Lists (有序列表)}

\subsubsection{功能介绍 (Introduction)} 

有序列表用于展示有顺序的步骤或排行。可以自动生成数字、字母或罗马数字编号。

Ordered lists are used to display sequential steps or rankings. They can automatically generate numbers, letters, or Roman numerals.

\vspace{0.5em}
\subsubsection{代码展示 (Code)}

\begin{minted}{latex}
\begin{enumerate}[label=\textbf{Step \arabic*.}]
    \item First step
    \item Second step
    \begin{enumerate}[label=(\alph*)]
        \item Sub-step A
        \item Sub-step B
    \end{enumerate}
\end{enumerate}
\end{minted}

\subsubsection{语法详解 (Syntax)}

\begin{itemize}
    \item \code{\textbackslash begin\{enumerate\}}: 开启有序列表环境。
    \item \code{label=\textbackslash arabic*}: 设置标签样式,\code{\textbackslash arabic*} 代表阿拉伯数字 (1, 2, 3)。
    \item \code{\textbackslash alph*}: 小写字母 (a, b, c)。
    \item \code{\textbackslash Roman*}: 大写罗马数字 (I, II, III)。
    \item \code{Step}: 可以在编号前后添加自定义文本。
\end{itemize}

\subsubsection{效果展示 (Effect)}

\begin{enumerate}[label=\textbf{Step \arabic*.}]
    \item First step
    \item Second step
    \begin{enumerate}[label=(\alph*)]
        \item Sub-step A
        \item Sub-step B
    \end{enumerate}
\end{enumerate}

% ----------------------------------------------------------

\subsection{Description Lists (描述列表)}

\subsubsection{功能介绍 (Introduction)} 

描述列表用于解释术语或定义概念,每个项目由一个加粗的标签和一段描述文本组成。

Description lists are used to explain terms or define concepts, where each item consists of a bold label and a description text.

\vspace{0.5em}
\subsubsection{代码展示 (Code)}

\begin{minted}{latex}
\begin{description}
    \item[LaTeX] A document preparation system.
    \item[CLI] Command Line Interface.
\end{description}
\end{minted}

\subsubsection{语法详解 (Syntax)}

\begin{itemize}
    \item \code{\textbackslash item[Label]}: 方括号内的内容会作为标签加粗显示。
\end{itemize}

\subsubsection{效果展示 (Effect)}

\begin{description}
    \item[LaTeX] A document preparation system used for high-quality typesetting.
    \item[CLI] Command Line Interface, interaction with computer programs via text.
\end{description}

% ----------------------------------------------------------

\subsection{Nested Lists (嵌套列表)}

\subsubsection{功能介绍 (Introduction)}

嵌套列表允许在列表项中创建子列表。LaTeX 自动为不同层级的列表处理标签样式(如:1 $\to$ (a) $\to$ i $\to$ A)。

Nested lists allow creating sub-lists within list items. LaTeX automatically handles label styles for different levels (e.g., 1 $\to$ (a) $\to$ i $\to$ A).

\vspace{0.5em}
\subsubsection{代码展示 (Code)}

\begin{minted}{latex}
\begin{enumerate}
    \item First top-level item
    \begin{enumerate}
        \item First sub-item
        \item Second sub-item
        \begin{enumerate}
            \item Deeply nested item
        \end{enumerate}
    \end{enumerate}
    \item Second top-level item
\end{enumerate}
\end{minted}

\subsubsection{语法详解 (Syntax)}

\begin{itemize}
    \item 嵌套列表最大深度默认为 4 层。
    \item \code{enumerate} 默认标签顺序:1, (a), i, A。
    \item \code{itemize} 默认标签顺序:$\bullet$, --, $\ast$, $\cdot$。
\end{itemize}

\subsubsection{效果展示 (Effect)}

\begin{enumerate}
    \item First top-level item
    \begin{enumerate}
        \item First sub-item
        \item Second sub-item
        \begin{enumerate}
            \item Deeply nested item
        \end{enumerate}
    \end{enumerate}
    \item Second top-level item
\end{enumerate}

% ----------------------------------------------------------

\subsection{Custom List Styles (自定义列表样式)}

\subsubsection{功能介绍 (Introduction)}

利用 \texttt{enumitem} 宏包,可以轻松修改列表的符号、编号样式、间距等。

Using the \texttt{enumitem} package, you can easily modify list symbols, numbering styles, spacing, etc.

\vspace{0.5em}
\subsubsection{代码展示 (Code)}

\begin{minted}{latex}
% Resume numbering from previous list
\begin{enumerate}[resume]
    \item Third item (continued)
\end{enumerate}

% Custom symbol and color
\begin{itemize}[label=\textcolor{blue}{\textbullet}]
    \item Blue bullet item
\end{itemize}

% Roman numerals with custom format
\begin{enumerate}[label=\Roman*., start=5]
    \item Item V
    \item Item VI
\end{enumerate}
\end{minted}

\subsubsection{语法详解 (Syntax)}

\begin{itemize}
    \item \code{resume}: 接续上前一个同类型列表的编号。
    \item \code{start=N}: 强制从数字 N 开始编号。
    \item \code{label=\textbackslash textcolor\{color\}\{symbol\}}: 自定义标签颜色和符号。
\end{itemize}

\subsubsection{效果展示 (Effect)}

\begin{enumerate}
    \item Previous item 1
    \item Previous item 2
\end{enumerate}

\begin{enumerate}[resume]
    \item Third item (continued numbering)
\end{enumerate}

\begin{itemize}[label=\textcolor{blue}{\textbullet}]
    \item Blue bullet item
\end{itemize}

\begin{enumerate}[label=\Roman*., start=5]
    \item Item V
    \item Item VI
\end{enumerate}

% ----------------------------------------------------------

\subsection{Inline Lists (行内列表)}

\subsubsection{功能介绍 (Introduction)}

行内列表用于在段落文字中穿插列表项,通常用于枚举短语。需要 \texttt{enumitem} 的 \texttt{inline} 选项(本模板已默认加载)。

Inline lists are used to intersperse list items within paragraph text, typically for enumerating phrases. Requires the \texttt{inline} option of \texttt{enumitem} (loaded by default in this template).

\vspace{0.5em}
\subsubsection{代码展示 (Code)}

\begin{minted}{latex}
Three things to remember:
\begin{enumerate*}[label=(\arabic*)]
    \item practice,
    \item practice, and
    \item more practice.
\end{enumerate*}
\end{minted}

\subsubsection{语法详解 (Syntax)}

\begin{itemize}
    \item \code{enumerate*}: 行内有序列表环境 (需 \texttt{enumitem[inline]})。
    \item \code{itemize*}: 行内无序列表环境。
    \item 适用于将列表项作为句子的一部分流畅展示。
\end{itemize}

\subsubsection{效果展示 (Effect)}

Three things to remember:
\begin{enumerate*}[label=(\arabic*)]
    \item practice,
    \item practice, and
    \item more practice.
\end{enumerate*}

% ----------------------------------------------------------

\subsection{Checklists (任务清单)}

\subsubsection{功能介绍 (Introduction)}

使用 \texttt{amssymb} 提供的符号创建待办事项清单。

Create to-do lists using symbols provided by \texttt{amssymb}.

\vspace{0.5em}
\subsubsection{代码展示 (Code)}

\begin{minted}{latex}
\begin{itemize}[label=$\square$]
    \item Unchecked task
    \item[$\boxtimes$] Checked task
    \item Another pending task
\end{itemize}
\end{minted}

\subsubsection{语法详解 (Syntax)}

\begin{itemize}
    \item \code{\textbackslash square}: 空方框符号 (需 \texttt{amssymb})。
    \item \code{\textbackslash boxtimes}: 带叉方框符号。
    \item \code{\textbackslash checkmark}: 对勾符号。
\end{itemize}

\subsubsection{效果展示 (Effect)}

\begin{itemize}[label=$\square$]
    \item Unchecked task
    \item[$\boxtimes$] Checked task
    \item Another pending task
\end{itemize}
