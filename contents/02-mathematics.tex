% ==========================================================
\section{Mathematics (数学公式)}
% ==========================================================

\subsection{Equations and Alignments (方程与对齐)}

\subsubsection{功能介绍 (Introduction)} 

LaTeX 最强大的功能之一是排版数学公式。我们可以使用 \texttt{equation} 环境进行自动编号,使用 \texttt{align} 环境进行多行公式对齐。

One of LaTeX's most powerful features is typesetting mathematics. We can use the \texttt{equation} environment for automatic numbering and the \texttt{align} environment for multi-line equation alignment.

\vspace{0.5em}
\subsubsection{代码展示 (Code)}

\begin{minted}{latex}
% Single numbered equation
\begin{equation}
    E = mc^2 \label{eq:einstein}
\end{equation}

% Aligned equations
\begin{align}
    f(x) &= (x+a)(x+b) \nonumber \\
         &= x^2 + (a+b)x + ab
\end{align}
\end{minted}

\subsubsection{语法详解 (Syntax)}

\begin{itemize}
    \item \code{equation}: 生成带编号的单行公式。
    \item \code{align}: 生成多行公式,使用 \code{\&} 指定对齐位置,反斜杠加反斜杠换行。
    \item \code{nonumber}: 阻止当前行生成编号。
    \item \code{label}: 添加引用标签。
\end{itemize}

\subsubsection{效果展示 (Effect)}

\begin{equation}
    E = mc^2 \label{eq:einstein}
\end{equation}

\begin{align}
    f(x) &= (x+a)(x+b) \nonumber \\
         &= x^2 + (a+b)x + ab
\end{align}

% ----------------------------------------------------------

\subsection{Matrices and Cases (矩阵与分段函数)}

\subsubsection{功能介绍 (Introduction)} 

矩阵和分段函数是线性代数和微积分中常见的结构。

Matrices and piecewise functions are common structures in linear algebra and calculus.

\vspace{0.5em}
\subsubsection{代码展示 (Code)}

\begin{minted}{latex}
$$
    \mathbf{A} = \begin{pmatrix} 
        1 & 0 \\ 
        0 & 1 
    \end{pmatrix}, 
    \quad
    |x| = \begin{cases}
        x & \text{if } x \ge 0 \\
        -x & \text{if } x < 0
    \end{cases}
$$
\end{minted}

\subsubsection{语法详解 (Syntax)}

\begin{itemize}
    \item \code{pmatrix}: 圆括号矩阵 \code{()}。
    \item \code{bmatrix}: 方括号矩阵 \code{[]}。
    \item \code{cases}: 分段函数环境,自动添加左大括号。
    \item \code{\textbackslash text}: 在数学模式中插入普通文本。
\end{itemize}

\subsubsection{效果展示 (Effect)}

$$
    \mathbf{A} = \begin{pmatrix} 
        1 & 0 \\ 
        0 & 1 
    \end{pmatrix}, 
    \quad
    |x| = \begin{cases}
        x & \text{if } x \ge 0 \\
        -x & \text{if } x < 0
    \end{cases}
$$

% ----------------------------------------------------------

\subsection{Calculus and Integrals (微积分与积分)}

\subsubsection{功能介绍 (Introduction)}

微积分公式通常包含极限、求和、积分等复杂的上下标结构。使用 \texttt{\$\$} 可以让这些公式独立成行显示。

Calculus formulas often involve complex limits, sums, and integrals with subscripts and superscripts. Using \texttt{\$\$} displays these formulas on their own lines.

\vspace{0.5em}
\subsubsection{代码展示 (Code)}

\begin{minted}{latex}
% Limit and Summation
$$ \lim_{x \to \infty} \frac{1}{x} = 0, \quad \sum_{n=1}^{\infty} \frac{1}{n^2} = \frac{\pi^2}{6} $$

% Definite Integral
$$ \int_{a}^{b} f(x) \,dx = F(b) - F(a) $$

% Triple Integral
$$ \iiint_{V} \rho(x,y,z) \,dx\,dy\,dz $$
\end{minted}

\subsubsection{语法详解 (Syntax)}

\begin{itemize}
    \item \code{\textbackslash lim}, \code{\textbackslash sum}, \code{\textbackslash int}: 分别表示极限、求和、积分。
    \item \code{\_\{...\}}: 下标 (Lower limit)。
    \item \code{\^\{...\}}: 上标 (Upper limit)。
    \item \code{\textbackslash to}: 趋近箭头 ($\to$)。
    \item \code{\textbackslash infty}: 无穷符号 ($\infty$)。
    \item \code{\textbackslash,}: 插入一个小间距 (thin space),常用于微分算子 $dx$ 前。
\end{itemize}

\subsubsection{效果展示 (Effect)}

$$ \lim_{x \to \infty} \frac{1}{x} = 0, \quad \sum_{n=1}^{\infty} \frac{1}{n^2} = \frac{\pi^2}{6} $$

$$ \int_{a}^{b} f(x) \,dx = F(b) - F(a) $$

$$ \iiint_{V} \rho(x,y,z) \,dx\,dy\,dz $$

% ----------------------------------------------------------

\subsection{Theorem Environments (定理环境)}

\subsubsection{功能介绍 (Introduction)}

\texttt{otex.sty} 预定义了常用的数学环境,如定理、定义、引理和证明。这些环境会自动处理编号和格式。

\texttt{otex.sty} predefines common mathematical environments such as theorems, definitions, lemmas, and proofs. These environments automatically handle numbering and formatting.

\vspace{0.5em}
\subsubsection{代码展示 (Code)}

\begin{minted}{latex}
\begin{definition}[Prime Number]
    A natural number greater than 1 that has no positive divisors other than 1 and itself.
\end{definition}

\begin{theorem}[Euclid]
    There are infinitely many prime numbers.
\end{theorem}

\begin{proof}
    Suppose there are finitely many primes $p_1, \dots, p_n$. Let $P = p_1 \cdots p_n + 1$. Then $P$ is not divisible by any $p_i$, contradiction.
\end{proof}
\end{minted}

\subsubsection{语法详解 (Syntax)}

\begin{itemize}
    \item \code{definition}: 定义环境。
    \item \code{theorem}: 定理环境,可选参数 \code{[Name]} 用于指定定理名称。
    \item \code{proof}: 证明环境,自动以 Q.E.D. 方块结尾。
    \item 更多可用环境: \code{lemma} (引理), \code{proposition} (命题), \code{corollary} (推论), \code{example} (例), \code{remark} (注).
\end{itemize}

\paragraph{配置定理名称 (Configuring Names)}

可以通过加载宏包时的选项或重定义宏来修改环境名称。

You can change the environment names by passing package options or redefining macros.

\begin{minted}{latex}
% Option 1: Use English names (Theorem, Definition, etc.)
% \usepackage[english]{otex}

% Option 2: Manually redefine specific names
\renewcommand{\algorithmcfname}{算法} % Rename algorithm
\renewcommand{\theoremname}{MyTheorem} % Custom theorem name
\end{minted}

\subsubsection{效果展示 (Effect)}

\begin{definition}[Prime Number]
    A natural number greater than 1 that has no positive divisors other than 1 and itself.
\end{definition}

\begin{theorem}[Euclid]
    There are infinitely many prime numbers.
\end{theorem}

\begin{proof}
    Suppose there are finitely many primes $p_1, \dots, p_n$. Let $P = p_1 \cdots p_n + 1$. Then $P$ is not divisible by any $p_i$, contradiction.
\end{proof}

% ----------------------------------------------------------

\subsection{Greek Letters and Special Symbols (希腊字母与特殊符号)}

\subsubsection{功能介绍 (Introduction)}

LaTeX 提供了大量的数学符号。以下是一些常用的希腊字母和关系符号的示例。

LaTeX provides a vast array of mathematical symbols. Below are examples of common Greek letters and relation symbols.

\vspace{0.5em}
\subsubsection{代码展示 (Code)}

\begin{minted}{latex}
$$
    \alpha, \beta, \gamma, \Delta, \pi, \theta, \phi
$$
$$
    \forall x \in \mathbb{R}, \exists y \text{ s.t. } y > x
$$
$$
    A \subseteq B, \quad A \cup B, \quad A \cap B, \quad x \neq y
$$
\end{minted}

\subsubsection{语法详解 (Syntax)}

\begin{itemize}
    \item 希腊字母: \code{\textbackslash alpha}, \code{\textbackslash beta}, \code{\textbackslash Delta} (大写), 等。
    \item 逻辑符号: \code{\textbackslash forall} ($\forall$), \code{\textbackslash exists} ($\exists$)。
    \item 集合符号: \code{\textbackslash in} ($\in$), \code{\textbackslash subseteq} ($\subseteq$), \code{\textbackslash cup} ($\cup$), \code{\textbackslash cap} ($\cap$)。
    \item \code{\textbackslash mathbb}: 黑板粗体,常用于表示数集 (需 \code{amssymb} 宏包)。
\end{itemize}

\subsubsection{效果展示 (Effect)}

$$
    \alpha, \beta, \gamma, \Delta, \pi, \theta, \phi
$$
$$
    \forall x \in \mathbb{R}, \exists y \text{ s.t. } y > x
$$
$$
    A \subseteq B, \quad A \cup B, \quad A \cap B, \quad x \neq y
$$

% ==========================================================
