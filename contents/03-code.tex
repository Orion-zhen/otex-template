% ==========================================================
\section{Code and Algorithms (代码与算法)}
% ==========================================================

\subsection{Code Listings (代码块)}

\subsubsection{功能介绍 (Introduction)} 

展示源代码时,通常需要语法高亮、行号和等宽字体。我们使用 \texttt{listings} 宏包来实现。

When displaying source code, syntax highlighting, line numbers, and monospaced fonts are usually required. We use the \texttt{listings} package to achieve this.

\vspace{0.5em}
\subsubsection{代码展示 (Code)}

\begin{lstlisting}[language=TeX]
\begin{lstlisting}[language=Python, caption={Python Example}, label={lst:py}]
def factorial(n):
    if n == 0:
        return 1
    return n * factorial(n-1)
\end{lstlisting }
\end{lstlisting} % Note: Space added in end tag to avoid nesting error

\subsubsection{语法详解 (Syntax)}

\begin{itemize}
    \item \code{language}: 指定代码语言 (如 Python, C++, Java)。
    \item \code{caption}: 添加代码块标题。
    \item \code{label}: 添加引用标签。
    \item 样式配置在 \texttt{otex.sty} 中定义。
\end{itemize}

\subsubsection{效果展示 (Effect)}

\begin{lstlisting}[language=Python, caption={Python Recursive Factorial}, label={lst:py}]
def factorial(n):
    """Calculate the factorial of n recursively."""
    if n == 0:
        return 1
    else:
        return n * factorial(n-1)
\end{lstlisting}

% ----------------------------------------------------------

\subsection{Algorithms (伪代码)}

\subsubsection{功能介绍 (Introduction)} 

算法伪代码提供了一种结构化的方式来描述算法逻辑,独立于具体编程语言。

Algorithm pseudocode provides a structured way to describe algorithmic logic, independent of specific programming languages.

\vspace{0.5em}
\subsubsection{代码展示 (Code)}

\begin{lstlisting}[language=TeX]
\begin{algorithm}[H]
    \SetAlgoLined
    \KwIn{List $L$}
    \KwOut{Sorted List $L$}
    \While{unsorted}{
        Find minimum element $m$\;
        Move $m$ to sorted portion\;
    }
    \caption{Selection Sort}
\end{algorithm}
\end{lstlisting}

\subsubsection{语法详解 (Syntax)}

\begin{itemize}
    \item \code{algorithm2e}: 使用的宏包。
    \item \code{\\KwIn}, \code{\\KwOut}: 输入输出关键字。
    \item \code{\\While}, \code{\\If}: 控制流结构。
    \item \code{\\;}: 语句结束的分号。
\end{itemize}

\subsubsection{效果展示 (Effect)}

\begin{algorithm}[H]
    \SetAlgoLined
    \KwIn{Array $A$ of $n$ elements}
    \KwOut{Sorted Array $A$}
    \For{$i \gets 0$ \KwTo $n-2$}{
        $min \gets i$\;
        \For{$j \gets i+1$ \KwTo $n-1$}{
            \If{$A[j] < A[min]$}{
                $min \gets j$\;
            }
        }
        Swap $A[i]$ and $A[min]$\;
    }
    \caption{Selection Sort Algorithm}
\end{algorithm}
