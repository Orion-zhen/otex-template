% ==========================================================
\section{Tables (表格)}
% ==========================================================

\subsection{Professional Tables (专业表格)}

\subsubsection{功能介绍 (Introduction)} 

科技论文中推荐使用所有的 ``三线表'',即仅使用顶部、中部和底部三条横线,去除垂直分割线。

Scientific papers recommend using "Three-line Tables", which use only top, middle, and bottom horizontal lines, removing vertical separators.

\vspace{0.5em}
\subsubsection{代码展示 (Code)}

\begin{minted}{latex}
\begin{table}[htbp]
    \centering
    \caption{Comparison of Metrics}
    \begin{tabular}{l c r}
        \toprule
        \textbf{Model} & \textbf{Accuracy} & \textbf{Time (s)} \\
        \midrule
        Baseline & 85.2\% & 12.5 \\
        Ours & 92.1\% & 15.3 \\
        \bottomrule
    \end{tabular}
\end{table}
\end{minted}

\subsubsection{语法详解 (Syntax)}

\textbf{Tabular 环境基础}:
\begin{minted}{latex}
\begin{tabular}{col_spec}
    Cell 1 & Cell 2 \\ 
    ...
\end{tabular}
\end{minted}

\begin{itemize}
    \item \textbf{列格式 (Column Specifiers)}: 定义在 \code{\begin{tabular}} 后的花括号中。
    \begin{itemize}
        \item \code{l}: 左对齐 (Left-aligned column)。
        \item \code{c}: 居中对齐 (Centered column)。
        \item \code{r}: 右对齐 (Right-aligned column)。
        \item \code{p{width}}: 指定宽度的段落列,内容会自动换行 (Paragraph column with fixed width, supports auto-wrapping)。
        \item \code{|}: 绘制垂直分割线 (Vertical line)。例如 \code{|c|c|} 表示两列都有边框。
    \end{itemize}
    \item \textbf{行与列分隔}:
    \begin{itemize}
        \item \code{\&}: 分隔同一行中的不同列 (Separates columns).
        \item \code{\\}: 结束当前行 (Ends the current row).
    \end{itemize}
    \item \code{booktabs}: 提供 \code{\toprule}, \code{\midrule}, \code{\bottomrule} 以绘制不同粗细的专业横线,替代标准的 \code{\hline}。
\end{itemize}

\subsubsection{效果展示 (Effect)}

\begin{table}[htbp]
    \centering
    \caption{Comparison of Accuracy and Performance}
    \begin{tabular}{l c r}
        \toprule
        \textbf{Model} & \textbf{Accuracy} & \textbf{Inference Time (ms)} \\
        \midrule
        ResNet-50 & 76.1\% & 24 \\
        ViT-Base & 81.3\% & 45 \\
        \textbf{Ours-Pro} & \textbf{83.5\%} & 38 \\
        \bottomrule
    \end{tabular}
\end{table}

\subsection{Column and Row Merging Fundamentals (行列合并基础)}

表格的高级功能核心在于“合并”。理解合并的关键在于:\textbf{表格本质上是一个网格,合并操作只是让某个单元格占据了邻近单元格的空间,但并未改变网格的底层逻辑。}

The core of advanced tables lies in "merging". The key is to understand that a table is a grid; merging simply allows a cell to occupy the space of its neighbors without changing the underlying grid logic.

\subsubsection{合并列 (\\multicolumn)}

合并列通过 \code{\multicolumn} 命令实现。它不仅用于合并,也常用于改变单个单元格的对齐方式(例如表头居中,数据居左)。

Column merging is done via \code{\multicolumn}. It is also used to change the alignment of a single cell.

\textbf{语法 (Syntax):}
\begin{minted}{latex}
\multicolumn{number}{align}{text}
\end{minted}
\begin{itemize}
    \item \code{number}: 要合并的列数 (Number of columns to merge)。
    \item \code{align}: 新的对齐方式 (e.g., \code{c}, \code{l}, \code{r}, \code{|c|})。
    \item \code{text}: 单元格内容。
\end{itemize}

\textbf{示例分析 (Example Analysis):}
我们想把第一行的两个单元格合并。

\begin{minted}{latex}
\begin{tabular}{|c|c|}
    \hline
    % 合并2列,居中对齐,内容为 "Title"
    \multicolumn{2}{|c|}{Merged Header} \\ 
    \hline
    Cell A & Cell B \\
    \hline
\end{tabular}
\end{minted}

\begin{center}
\begin{tabular}{|c|c|}
    \hline
    \multicolumn{2}{|c|}{\textbf{Merged Header}} \\ 
    \hline
    Cell A & Cell B \\
    \hline
\end{tabular}
\end{center}

\subsubsection{合并行 (\\multirow)}

合并行需要 \code{multirow} 宏包,命令为 \code{\multirow}。

Row merging requires the \code{multirow} package.

\textbf{语法 (Syntax):}
\begin{minted}{latex}
\multirow{number}{width}{text}
\end{minted}
\begin{itemize}
    \item \code{number}: 要合并的行数。
    \item \code{width}: 单元格宽度,通常填 \code{*} 让其自动适应。
    \item \code{text}: 单元格内容。
\end{itemize}

\textbf{重要规则 (Crucial Rule):}
当使用了 \code{\multirow{2}{...}} 时,当前行占据了位置。\textbf{在下一行对应的位置必须留空}(即直接写 \code{\&} 而不填内容),否则内容会重叠。

When using \code{\\multirow}, the corresponding cell in the \textbf{next rows} must be left empty.

\textbf{示例分析 (Example Analysis):}

\begin{minted}{latex}
\begin{tabular}{|c|c|}
    \hline
    % 第1行:声明合并2行,内容 "Common"
    \multirow{2}{*}{Common} & Row 1 Data \\ \cline{2-2}
    % 第2行:第一列留空!因为被上面占用了
     & Row 2 Data \\ 
    \hline
\end{tabular}
\end{minted}

\begin{center}
\begin{tabular}{|c|c|}
    \hline
    \multirow{2}{*}{Common} & Row 1 Data \\ \cline{2-2}
     & Row 2 Data \\ 
    \hline
\end{tabular}
\end{center}

注意使用 \code{\cline{2-2}} 来只画第2列的横线,否则会穿过 "Common" 文本。

Notice using \code{\cline{2-2}} to draw a horizontal line only across the 2nd column.

\subsection{Step-by-Step Complex Table (手把手构建复杂表格)}

让我们构建一个复杂的“课程表”,包含多级表头。

Let's build a complex "Schedule" with multi-level headers.

\textbf{目标结构 (Goal Logic):}
\begin{itemize}
    \item 总共4列。
    \item 表头第1列是 "Day" (需要跨越3行: Header 1, 2, 3)。
    \item 表头第2-4列是 "Schedule" (跨越3列)。
    \item "Schedule" 下面分为 "Morning" (跨越1列, 跨越2行) 和 "Afternoon" (跨越2列)。
    \item "Afternoon" 下面分为 "13:00" 和 "15:00"。
\end{itemize}

\textbf{代码实现 (Implementation):}

\begin{minted}{latex}
\begin{table}[htbp]
\centering
\caption{Complex Schedule Construction}
\begin{tabular}{l c c c}
    \toprule
    % --- Row 1 ---
    % Col 1: Day (跨3行)
    % Col 2-4: Schedule (跨3列)
    \multirow{3}{*}{\textbf{Day}} & \multicolumn{3}{c}{\textbf{Schedule}} \\ 
    % 绘制横线: 从第2列到第4列 (不穿过Day)
    \cmidrule(lr){2-4}
    
    % --- Row 2 ---
    % Col 1: 空 (Day占位)
    % Col 2: Morning (跨2行)
    % Col 3-4: Afternoon (跨2列)
     & \multirow{2}{*}{\textbf{Morning}} & \multicolumn{2}{c}{\textbf{Afternoon}} \\ 
    % 绘制横线: 从第3列到第4列 (Morning下方不画线)
    \cmidrule(lr){3-4}
    
    % --- Row 3 ---
    % Col 1: 空 (Day占位)
    % Col 2: 空 (Morning占位)
    % Col 3: 13:00
    % Col 4: 15:00
     & & \textbf{13:00} & \textbf{15:00} \\ 
    \midrule
    
    % --- Data Rows ---
    Mon & Math & Physics & History \\
    Tue & CS & \multicolumn{2}{c}{Lab (3 hours)} \\ % 数据也可以合并
    \bottomrule
\end{tabular}
\end{table}
\end{minted}

\subsubsection{最终效果 (Final Effect)}

\begin{table}[htbp]
\centering
\caption{Complex Schedule Construction}
\begin{tabular}{l c c c}
    \toprule
    \multirow{3}{*}{\textbf{Day}} & \multicolumn{3}{c}{\textbf{Schedule}} \\ 
    \cmidrule(lr){2-4}
     & \multirow{2}{*}{\textbf{Morning}} & \multicolumn{2}{c}{\textbf{Afternoon}} \\ 
    \cmidrule(lr){3-4}
     & & \textbf{13:00} & \textbf{15:00} \\ 
    \midrule
    Mon & Math & Physics & History \\
    Tue & CS & \multicolumn{2}{c}{Lab (3 hours)} \\
    \bottomrule
\end{tabular}
\end{table}

\subsection{Colored Tables (彩色表格)}

\subsubsection{功能介绍 (Introduction)}
通过 \code{xcolor} 宏包的 \code{table} 选项,可以轻松为表格添加背景色。

\vspace{0.5em}
\subsubsection{代码展示 (Code)}
\begin{minted}{latex}
% \usepackage[table]{xcolor}
\begin{tabular}{l c}
    \toprule
    % 行颜色
    \rowcolor{blue!10} \textbf{Item} & \textbf{Cost} \\
    \midrule
    Apple & \$1.00 \\
    % 单元格颜色覆盖行颜色
    Banana & \cellcolor{red!10} \$0.80 \\
    \bottomrule
\end{tabular}
\end{minted}

\subsubsection{颜色格式详解 (Color Syntax)}
\code{xcolor} 提供了强大的颜色混合语法。

\begin{itemize}
    \item \textbf{基础颜色}: \code{red}, \code{blue}, \code{green}, \code{black}, \code{white}, \code{yellow}, \code{cyan}, \code{magenta}.
    \item \textbf{透明度/混合 (Mixing)}: 
    \begin{itemize}
        \item \code{blue!40}: 40\% 蓝色 + 60\% 白色 (40\% Blue, 60\% White).
        \item \code{red!30!black}: 30\% 红色 + 70\% 黑色 (30\% Red mixed with Black).
        \item \code{blue!50!green}: 50\% 蓝色 + 50\% 绿色 (Mix Blue and Green).
    \end{itemize}
    \item \textbf{自定义颜色 (Custom Colors)}:
    \begin{minted}{latex}
\definecolor{myblue}{RGB}{0, 100, 200}    % 0-255 RGB
\definecolor{myred}{HTML}{FF0000}         % Hex
\definecolor{mygray}{gray}{0.9}           % 0-1 Grayscale
    \end{minted}
\end{itemize}

\subsubsection{效果展示 (Effect)}
\begin{table}[htbp]
    \centering
    \caption{Colorful Table Example}
    \begin{tabular}{l c}
        \toprule
        \rowcolor{blue!10} \textbf{Item} & \textbf{Cost} \\
        \midrule
        Apple & \$1.00 \\
        Banana & \cellcolor{red!10} \$0.80 \\
        \bottomrule
    \end{tabular}
\end{table}

\subsection{Long Tables (跨页长表格)}

\subsubsection{功能介绍 (Introduction)}
对于超过一页的表格,使用 \code{longtable} 环境。

\begin{minted}{latex}
\begin{longtable}{l p{8cm}}
    \caption{Long Table} \\
    \toprule
    Header 1 & Header 2 \\
    \midrule
    \endfirsthead
    % ... (后续页表头定义) ...
    \endhead
    % ... (页脚定义) ...
    \endfoot
    % ... (最后页脚定义) ...
    \endlastfoot
    
    Row 1 & Data... \\
    Row 2 & Data... \\
\end{longtable}
\end{minted}

\subsubsection{效果展示 (Effect)}
\begin{longtable}{l p{8cm}}
    \caption{Long Table Example} \\
    \toprule
    \textbf{Header 1} & \textbf{Header 2} \\
    \midrule
    \endfirsthead
    \multicolumn{2}{c}%
    {{\bfseries \tablename\ \thetable{} -- continued}} \\
    \toprule
    \textbf{Header 1} & \textbf{Header 2} \\
    \midrule
    \endhead
    \midrule
    \multicolumn{2}{r}{{Continued...}} \\
    \bottomrule
    \endfoot
    \bottomrule
    \endlastfoot
    
    Row 1 & Data for row 1 which might be long. \\
    Row 2 & Data for row 2 which might be long. \\
    Row 3 & Data for row 3 which might be long. \\
    Row 4 & Data for row 4 which might be long. \\
    Row 5 & Data for row 5 which might be long. \\
    Row 6 & Data for row 6 which might be long. \\
\end{longtable}
